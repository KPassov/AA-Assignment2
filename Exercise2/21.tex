\subsection*{2.1}
We define our z values as seen on tables \ref{tab:zgen} and \ref{tab:zs}
\begin{table}[ht!]
    \resizebox{\columnwidth}{!}{
    \centering
    $\begin{array}{l||c|c|c|c|c}
          & a                           &   b       & c         & d                 & e       \\\hline\hline
        a & -                           & (v1,v6)   & (v1,v3)   & NA                & (v3,v5) + (v5,v7) + (v6,v7) \\
        b & (v6,v1)                     & -         & (v1,v2)   & (v2,v6)           & NA      \\
        c & (v3,v1)                     & (v2,v1)   & -         & (v2,v3)           & NA      \\
        d & NA                          & (v6,v2)   & (v3,v2)   & -                 & (v3,v4) + (v4,v6) \\
        e & (v7,v6) + (v7,v5) + (v5,v3) & NA        & NA        & (v6,v4) + (v4,v3) & -       \\
    \end{array}$
    } 
    \caption{The edges between each pair of bounded cycles}
    \label{tab:zgen}
\end{table}
\begin{table}[ht!]
    \[ \begin{array}{l||c|c|c|c|c}
          & a & b & c & d & e \\\hline\hline
        a & 0 & 0 & 0 & 0 & 0 \\
        b & 2 & 0 & 1 & 1 & 0 \\
        c & 1 & 1 & 0 & 0 & 0 \\
        d & 0 & 1 & 0 & 0 & 2 \\
        e & 4 & 0 & 0 & 0 & 0 \\
    \end{array} 
    \]
    \caption{The actual number of breakpoints between each pair of bounded cycles}
    \label{tab:zs}
\end{table}
\begin{equation}\label{eq:bp}
    \sum\limits_{g}\sum\limits_{f} z_{gf} = \text{Total breakpoints}
\end{equation}
We find the total breakpoints in all edges by (\ref{eq:bp}) and get \textbf{13}.
\begin{table}[!ht]
    \[ \begin{array}{l||c|c|c|c|c|c|c}
          & v_1 & v_2 & v_3 & v_4 & v_5 & v_6 & v_7 \\\hline\hline
        a & 0   &     & 1   &     & 1   & 1   & 0   \\
        b & 1   & 0   &     &     &     & 1   &     \\
        c & 1   & 1   & 1   &     &     &     &     \\
        d &     & 1   & 1   & -1  &     & 1   &     \\
        e &     &     & 1   & 1   & -1  & 1   & 0   \\
    \end{array} \]
\end{table}
