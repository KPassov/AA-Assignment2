\begin{proof}
    We want to show that the following is true using lemma 1.
    \begin{equation*}
        Pr[X_C < (1-ab)\mu_C] \le \frac{1}{r^2}\\
    \end{equation*}

    So we did some substitutions where our goal was to end up with lemma 1:
    \begin{alignat*}{2}
        Pr[&X_C < (1+b)\mu_C]& \le \frac{1}{r^2}\\
        &X_C < \mu_C + \mu_C b&\\
        &X_C - \mu_C < \mu_C b&\\
        Pr[&X_C - \mu_C < \mu_C r \frac{1}{\sqrt{k}}]& \le \frac{1}{r^2}\\ 
    \end{alignat*}
    So we wanted to end up with this:
    \begin{equation*}
        Pr[|X_C - \mu_C| > r \sqrt{\mu}] \le \frac{1}{r^2}
    \end{equation*}

    But what we have is not completely it as we have:
    \begin{align*}
        r \sqrt{\mu} &> \frac{\sqrt{fk}}{1-b} r\\
        |X_A - \mu_A| &> \frac{\sqrt{fk}}{1-a} r
    \end{align*}
    and therefore we can not say:
    \begin{equation*}
        |X_A - \mu_C| > r \sqrt{\mu_C}
    \end{equation*}

    We are not really sure where to go from here, but we think we are on the
    right track.


\end{proof}
